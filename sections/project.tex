\cvsection{Course Projects}

\begin{cventries}


\begin{comment}
   \cventry
  {Computer Networks, Course Assignments}
  {Implementation of Automatic Repeat Request(ARQ) Protocols}
  {IIT kanpur}
  {Jan,2020 - May,2021}
  {
     \begin{cvitems}
      \item Implemented Stop and Wait ARQ protocol and  Go-Back-N ARQ protocol with proper acknowledgements.
      \item Written code for both sender and receiver in \textbf{python} to transfer message/files using socket programming.
     \end{cvitems}
  }
\end{comment}

   \cventry
  {Analysis of Concurrent Programs, Course Assignments}
  {Implementing modified FastTrack and Concurrent Data Structures}
  {IIT kanpur}
  {Jan,2021 - May,2021}
  {
     \begin{cvitems}
      \item Implemented sampling support, schedule steering and perturbation in race detector tool FastTrack
      \item Implemented concurrent data structure such as \textbf{lock-free MS Queue} and a \textbf{non-blocking} version of hash table
     \end{cvitems}
  }

% \begin{comment}
   \cventry
  {Embedded and Cyber-Physical Systems, Course Project}
  {Simulated an Inverted-Pendulumn - a CPS System}
  {IIT kanpur}
  {Sep,2020 - Dec,2020}
  {
     \begin{cvitems}
      \item Simulated stabilising of an inverted pendulum, a cyber-physical system, on low power multi-hop network on simulink
      \item Experimented with the simulated system by varying jitter, delay, phase in the signal to the actuators
     \end{cvitems}
  }
% \end{comment}

%   \cventry
%   {Computer Architecture, Course Project}
%   {Implementation of RDIP: Return-address-stack Directed Instruction Prefetching}
%   {IIT kanpur}
%   {Feb,2020 - April,2020}
%   {
%      \begin{cvitems}
%       \item RDIP is a method to implement instruction prefetching on basis of RAS Return address Stack of a program.
%       \item \textbf{Written L1 instruction prefetcher code in C}, based on Return-address-stack of Programs, on ChampSim. 
%       \item The \textbf{benchmarks} used for testing was \textbf{ISCA IPC1 traces}.
%       \item \textbf{Experinced 22\% instruction prefetching hit} which is greater than the next two line instruction prefetcher of 8\% hit rate.
%      \end{cvitems}
%   }

\begin{comment}
  \cventry
  {Compilers, Course Project}
  {JAVA lexer and Parser}
  {IIT kanpur}
  {Jan,2020 - March,2020}
  {
     \begin{cvitems}
      \item Learned and used lex (lexical analyzer), YACC and ANTLR (Parser).
      \item Used lex for lexical analysis for JAVA8 and also ANTLR(LL* Parser) for lexing and parsing JAVA8 code.
      \item Made Abstract Syntax Tree of JAVA code using graphviz.
     \end{cvitems}
  }

  \cventry
  {Operating Systems, Course Work}
  {Learning Intricacies of OS using GemOS}
  {IIT kanpur}
  {Aug,2019 - Nov,2019}
  {
     \begin{cvitems}
      \item Implemented and tested various operating system design such as paging, context switch, file system and multithreading:-
      \item \textbf{file system syscall} - open(), read(), write(), fork(), dup() etc.
      \item \textbf{paging management} using syscall like mmap(), munmap(), mprotect(), cfork() and vfork().
      \item \textbf{multithreaded hash table using locks and semaphores} for preventing simultaneous access. 
     \end{cvitems}
  }
\end{comment}
  \cventry
  {Recommendation System \& DeCaptcha, Course Assignments}
  {Machine Learning}
  {IIT kanpur}
  {Aug,2019 - Nov,2019}
  {
     \begin{cvitems}
      \item Made a \textbf{recommendation system} (multi-label classification) using a \textbf{tree+ classifier} for prediction and beam search (Bonsai)
      \item Build a \textbf{CNN model using keras and tensorflow} libraries to identify the characters in a given Captcha 
      \item Model consists of 2 hidden layers, 1 input layer containing 10,000 input nodes and 1 output layer containing 26 output nodes
      \item Used "ReLu" and "softmax" as activation function in the hidden layers 
     \end{cvitems}
  }

  \cventry
  {Computing Laboratory, Course Assignments}
  {Android App}
  {IIT kanpur}
  {Aug,2019 - Nov,2019}
  {
     \begin{cvitems}
      \item Built a native application using \textbf{react native, mongoDB and Nodejs} where user can signup and share there thoughts 
     \end{cvitems}
  }

%   \cventry
%   {under Prof. Debadatta Mishra, Dept. of CSE}
%   {System Visualiser}
%   {IIT kanpur}
%   {May,2019 - July,2019}
%   {
%      \begin{cvitems}
%       \item Created an application in C++ using \textbf{QtCreator}, which \textbf{visualises the computer instructions}, in the form of tables and animations, given by gemOS(a teaching OS)
%       running on gem5 architectural simulator using the log files generated.
%       % \item Interpreted the large log files to create meaningful information in the form of tables and animations to create the visualization.
%      \end{cvitems}
%   }

\begin{comment}
  \cventry 
  {Association of Computing Activities, IITK}
  {Cyber Security}
  {IIT Kanpur}
  {Jan,2018 - April,2018}
  {
    \begin{cvitems}
      \item Solved several CTF(capture the flag contests) in which a hidden flag (keyword) is to  be found in a compiled C file by using assembly language and GDB (General Debugger) in Linux. 
      \item Learned to alter the flow of a C program, hence able to access specific memory address/hidden functions and surpass basic security features.
    \end{cvitems}
  }
  \cventry
  {To learn new concepts}
  {Self-Projects}
  {}
  {}
  {
    \begin{cvitems}
      \item App Development - Developed a simple app which could perform basic functions of a calculator in Android Studio
      \item Sudoku Solver/Generator -  Build a program in C++ which can solve sudoku problems and can generate sudokus with a
unique solution and a minimum number of hints using MiniSAT
    \end{cvitems}
  }
\end{comment}
\end{cventries}

%%% Local Variables:
%%% mode: latex
%%% End:
